\documentclass[12pt]{article}
\usepackage[T2A]{fontenc}
\usepackage[utf8]{inputenc}
\usepackage[russian]{babel}
\usepackage[normalem]{ulem}
\usepackage{amsmath}
\usepackage{amssymb}
\usepackage{amsthm}
\usepackage{blindtext}
\usepackage{color}
\usepackage{geometry}
\usepackage{graphicx}
\usepackage{hologo}
\usepackage{listings}
\usepackage{mathtools}
\usepackage{hyperref}
\usepackage{nccmath}
\usepackage{xcolor}

\setcounter{secnumdepth}{-1}


\geometry{
    a4paper,
    total={527pt, 770pt},
    left=35pt,
    top=35pt
}
% \input code-format.tex


\begin{document}

\begin{center}
\textbf{Подробности передачи ресурса с удаленного устройства методом, использующим Content-addressable storage}
\end{center}

Как \texttt{desync}, так и \texttt{casync} поддерживают, кроме всего прочего, операции двух типов: деления ресурса на чанки (make и tar для \texttt{desync}, make для \texttt{casync}) и доставки (extract и untar для \texttt{desync}, extract для \texttt{casync}). При этом методы доставки позволяют указать место хранения кеша на локальном устройстве и удаленное хранилище чанков. То есть передавая в команду доставки аргументом \texttt{caibx/caidx}-файл (индексный файл, указывающий на файл и директорию, соотвественно), необходимые чанки будут сначала искаться в локальном хранилище, а только после этого - в удаленном. Таким образом, скачиваться будут только необходимые куски.

То есть использование утилит \texttt{desync} и \texttt{casync} состоит из двух этапов: разделение исходного ресурса на чанки и загрузка в удаленный репозиторий индексного файла и этих чанков (производится 1 раз для всех клиентов), далее на каждом клиенте вызывается метод доставки, указывая при этом индексный файл, локальное и удаленное хранилища кеша.

\begin{center}
\textbf{Экспериментальная установка}
\end{center}

Рассматриваются версии пакета $k, k + 1, k + 2, \dots, k + m$. Хотим эмпирически измерить время доставки $(k + m)-$й версии при условии:

1) Если способ доставки использует технологию Content-addressable storage, то предварительно клиент хранит кеш от версий $k, k + 1, \dots, k + m - 1$.

2) Иначе $(k + m)-$я версия замещает $(k + m - 1)$-ю.

Построим для каждого рассматриваемого ресурса графики времени доставки от номера версии $m$.

Рассматриваемые ресурсы:

1. Директории с кодом проекта

В качестве проекта возьмем \href{https://github.com/ClickHouse/ClickHouse}{ClickHouse}.

Как версии проекта возьмем версию репозитория с разницей в месяц, начиная с  июня 2024 года. Всего получается 11 версий. Размер одной версии $\approx 220M$.

2. Бинарный файл с данными 


В качестве ресурса возьмем \href{https://github.com/neovim/neovim/}{neovim} версий 7.0-10.2, получилось 11 версий. Размер одной версии $\approx 39M$.

\end{document}
